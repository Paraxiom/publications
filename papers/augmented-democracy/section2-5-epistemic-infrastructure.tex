\section{Procedural Infrastructure: Artifact Curation and Credential Management}
\label{sec:epistemic}

The ``augmented'' in augmented democracy refers not to technological enhancement of voting
mechanics, but to \textit{procedural augmentation}---ensuring that participants have
verifiably engaged with relevant evidence artifacts before their input shapes collective decisions.
This section describes the infrastructure for artifact curation, credential management,
and participation constraints.

\subsection{The Engagement Problem in Democratic Systems}

Classical democratic theory assumes informed voters. Reality provides participants who
may not have engaged with relevant evidence. The augmented democracy framework
addresses this through \textit{mandatory admissibility gates}---structural requirements
that participants demonstrate engagement with proposal-relevant artifacts before voting.

\begin{definition}[Admissibility Gate]
A procedural checkpoint $\mathcal{A}: \mathcal{V} \rightarrow \{0, 1\}$ that maps
a potential voter $v \in \mathcal{V}$ to eligibility status based on demonstrated
engagement with admissible artifacts for the proposal under consideration.
\end{definition}

\textbf{Critical clarification}: This is not a ``literacy test'' or competency filter.
It verifies \textit{engagement with admissible artifacts}, not intelligence, education,
political alignment, or agreement with artifact conclusions. A voter who has engaged
with the relevant evidence and disagrees with its conclusions still passes the gate.

\subsection{Token Curated Test Grids}

The admissibility gate mechanism is implemented through \textbf{Token Curated Test Grids}
(TCTGs), a structure adapted from token curated registries with economic incentives
for quality curation.

As established in Section~\ref{sec:philosophy}, test grids govern \textit{admissibility
criteria}---which artifact classes may be referenced for a proposal domain---not
semantic conclusions or truth claims.

\subsubsection{The Tripartite Structure}

Following the KILT Protocol model~\cite{kilt2020}, TCTGs employ three distinct roles:

\begin{enumerate}
    \item \textbf{Claimer}: A participant who asserts readiness to vote on a proposal
    \item \textbf{Attester}: A curator who has assembled the test grid for that proposal
    \item \textbf{Verifier}: The system (or designated validators) that checks test completion
\end{enumerate}

This separation prevents conflicts of interest: those who create tests do not
administer them, and those who verify do not profit from outcomes.

\subsubsection{Gatherers and Curators}

Within the Attester role, two sub-functions operate:

\begin{itemize}
    \item \textbf{Gatherers}: Identify admissible artifacts for a proposal---documents
    with valid provenance from recognized issuers (scientific journals, government
    agencies, standards bodies). Gatherers are compensated per artifact that meets
    admissibility criteria.

    \item \textbf{Curators}: Assemble gathered artifacts into test grids that verify
    engagement. The goal is to confirm that voters have \textit{encountered} the
    relevant evidence, not that they agree with it. Curators stake tokens on grid quality.
\end{itemize}

\begin{figure}[h]
\centering
\begin{tikzpicture}[node distance=1.5cm, auto]
    \node[draw, rectangle] (proposal) {Proposal Submitted};
    \node[draw, rectangle, below of=proposal] (gatherers) {Gatherers Identify Artifacts};
    \node[draw, rectangle, below of=gatherers] (curators) {Curators Assemble Grid};
    \node[draw, rectangle, below of=curators] (test) {Test Grid Published};
    \node[draw, rectangle, below of=test] (voter) {Voter Demonstrates Engagement};
    \node[draw, diamond, below of=voter, aspect=2, node distance=2cm] (pass) {Pass?};
    \node[draw, rectangle, left of=pass, node distance=3cm] (eligible) {Vote Eligible};
    \node[draw, rectangle, right of=pass, node distance=3cm] (retry) {Review \& Retry};

    \draw[->] (proposal) -- (gatherers);
    \draw[->] (gatherers) -- (curators);
    \draw[->] (curators) -- (test);
    \draw[->] (test) -- (voter);
    \draw[->] (voter) -- (pass);
    \draw[->] (pass) -- node[above] {Yes} (eligible);
    \draw[->] (pass) -- node[above] {No} (retry);
    \draw[->] (retry) |- (voter);
\end{tikzpicture}
\caption{Token Curated Test Grid Flow}
\label{fig:tctg-flow}
\end{figure}

\subsubsection{Economic Incentives and Slashing Conditions}

Token holders curating test grids face a strategic tension:

\begin{quote}
``Token holders have a tactical incentive to challenge and reject every candidate
to their registry. In the interest of increasing their holdings, this is at odds
with their strategic interest of increasing the value of their holdings. An empty
list is of no interest to consumers.'' --- KILT Protocol
\end{quote}

Applied to TCTGs: curators who create impossible tests drive away voters, reducing
the value of the governance token. Curators who create trivial tests undermine
procedural quality, also reducing token value. The equilibrium favors \textit{fair
tests that accurately verify artifact engagement}.

\textbf{Slashing conditions} (curators lose staked tokens):
\begin{itemize}
    \item Admitting artifacts with invalid provenance (forged signatures, broken hashes)
    \item Admitting artifacts from non-recognized issuers
    \item Admitting artifacts outside the declared artifact class for the proposal domain
    \item Admitting artifacts that were retracted by their issuing authority
\end{itemize}

\textbf{Not slashable} (curators are protected):
\begin{itemize}
    \item Admitting artifacts whose conclusions are later contested or revised
    \item Admitting artifacts that some participants disagree with
    \item Admitting artifacts from one scientific position when others exist
\end{itemize}

The system does not adjudicate semantic disputes. Curators are accountable for
\textit{procedural validity}, not \textit{correctness}.

\subsection{Dynamic NFTs for Credential Management}

Participant credentials in augmented democracy are not static. A voter's eligibility,
weight, and privileges evolve based on contribution history. This is implemented
through \textbf{Dynamic NFTs}---non-fungible tokens whose metadata updates based
on on-chain activity.

\subsubsection{Credential Lifecycle}

\begin{lstlisting}[language=Rust, caption={Dynamic Credential Structure}]
pub struct DynamicCredential<AccountId> {
    pub holder: AccountId,
    pub credential_type: CredentialType,
    pub issued_at: u64,
    pub last_updated: u64,

    // Dynamic fields (updated on-chain)
    pub reputation_score: u64,
    pub proposals_voted: u32,
    pub engagements_verified: u32,  // Test grids passed
    pub engagements_failed: u32,
    pub contributions: u32,
    pub domains_certified: BoundedVec<DomainId, MaxDomains>,

    // Computed eligibility
    pub voting_weight_multiplier: u64,  // basis points
    pub can_submit_proposals: bool,
    pub can_curate_grids: bool,
}

pub enum CredentialType {
    Citizen,        // Basic participation rights
    Contributor,    // Has passed contribution threshold
    Curator,        // Can assemble test grids
    Validator,      // Can verify consensus
    Guardian,       // Emergency governance rights
}
\end{lstlisting}

The credential NFT updates automatically when:
\begin{itemize}
    \item An engagement verification is passed or failed
    \item A vote is cast
    \item A proposal is submitted
    \item Reputation is adjusted by peer review
    \item Domain certification is earned or revoked
\end{itemize}

\subsubsection{Life-Sustaining NFTs}

A critical innovation is the \textbf{Life-Sustaining NFT}---a credential that requires
ongoing activity to remain valid. Unlike static credentials that persist indefinitely,
life-sustaining NFTs decay without continuous participation.

\begin{lstlisting}[language=Rust, caption={Life-Sustaining Credential Logic}]
pub struct LifeSustainingCredential<BlockNumber> {
    pub base_credential: DynamicCredential,
    pub vitality: u64,              // Current life points
    pub max_vitality: u64,          // Maximum life points
    pub decay_rate: u64,            // Points lost per epoch
    pub last_activity: BlockNumber, // Last qualifying action
    pub revival_cost: Balance,      // Cost to revive if expired
}

impl LifeSustainingCredential {
    pub fn is_alive(&self, current_block: BlockNumber) -> bool {
        let epochs_elapsed = (current_block - self.last_activity)
            / EPOCH_LENGTH;
        let decay = epochs_elapsed * self.decay_rate;
        self.vitality > decay
    }

    pub fn sustain(&mut self, activity_points: u64) {
        self.vitality = min(
            self.vitality + activity_points,
            self.max_vitality
        );
        self.last_activity = current_block();
    }
}
\end{lstlisting}

Life-sustaining NFTs address the ``ghost voter'' problem: credentials issued to
participants who subsequently disengage. By requiring periodic activity (voting,
contributing, engagement verification), the system ensures that voting weight reflects
\textit{active} participation, not historical registration.

\subsubsection{Domain-Specific Credentials}

Voters may hold credentials in specific domains:

\begin{center}
\begin{tabular}{ll}
\toprule
\textbf{Domain} & \textbf{Unlocks Voting On} \\
\midrule
Environmental & Climate, conservation, pollution proposals \\
Technical & Infrastructure, protocol upgrades \\
Economic & Treasury, tokenomics, funding proposals \\
Social & Community guidelines, dispute resolution \\
Emergency & Crisis response, security incidents \\
\bottomrule
\end{tabular}
\end{center}

Domain credentials are earned by passing domain-specific engagement verifications and maintained
through ongoing participation in that domain. A participant may hold multiple
domain credentials, each with independent vitality.

\subsection{Quadratic Voting for Bounded Influence}

The augmented democracy framework incorporates \textbf{quadratic voting} to prevent
plutocratic capture while preserving signal strength for high-conviction preferences.

\subsubsection{The Quadratic Cost Function}

The cost to cast $n$ votes on a single proposal from a single participant:

\begin{equation}
    \text{cost}(n) = n^2
\end{equation}

\begin{center}
\begin{tabular}{ccc}
\toprule
\textbf{Votes Cast} & \textbf{Cost} & \textbf{Marginal Cost} \\
\midrule
1 & 1 & 1 \\
2 & 4 & 3 \\
3 & 9 & 5 \\
4 & 16 & 7 \\
5 & 25 & 9 \\
\bottomrule
\end{tabular}
\end{center}

The increasing marginal cost discourages concentration of voting power on single
proposals, encouraging participants to distribute influence across multiple issues.

\subsubsection{Integration with Reputation Weighting}

Quadratic voting combines with reputation-based weighting:

\begin{equation}
    w_{\text{effective}} = \sqrt{\text{votes\_purchased}} \times r_i \times (1 + \epsilon_i)
\end{equation}

where $r_i$ is reputation score and $\epsilon_i$ is the quantum entropy adjustment
from Section~\ref{sec:governance-control}. The square root of purchased votes
ensures diminishing returns, while reputation and entropy preserve the coherence
mechanisms.

\subsubsection{Whale Resistance}

Consider an adversary with 100$\times$ the resources of an average participant:

\begin{center}
\begin{tabular}{lcc}
\toprule
\textbf{System} & \textbf{Adversary Influence} & \textbf{Ratio} \\
\midrule
1-person-1-vote & 1 vote & 1:1 \\
Plutocratic (1:1 stake) & 100 votes & 100:1 \\
Quadratic & 10 votes & 10:1 \\
Quadratic + Reputation Cap & $\leq$ 10 votes & $\leq$ 10:1 \\
\bottomrule
\end{tabular}
\end{center}

Quadratic voting reduces the 100:1 wealth advantage to a 10:1 voting advantage.
Combined with reputation caps and coherence thresholds, adversarial influence
is further bounded.

\subsection{Deliberative Democracy: Multi-Round Consensus}

For high-stakes proposals, the framework supports \textbf{deliberative democracy}
through multiple voting rounds with increasing consensus requirements.

\subsubsection{Unanimity-Seeking Processes}

Drawing from North American Indigenous governance traditions:

\begin{quote}
``Unanimity requires that everyone involved agrees.''
\end{quote}

While perfect unanimity is impractical at scale, the framework supports
\textit{unanimity-seeking} processes:

\begin{enumerate}
    \item \textbf{Round 1}: Simple majority required
    \item \textbf{Round 2}: If Round 1 passes but dissent exceeds threshold,
    deliberation period opens; 60\% supermajority required
    \item \textbf{Round 3}: If significant dissent remains, face-to-face
    (or synchronous digital) deliberation; 75\% required
    \item \textbf{Final Round}: Consensus conference with all registered
    dissenters; 90\% required or proposal modified
\end{enumerate}

This process is expensive and slow---by design. It applies only to proposals
flagged as ``constitutional'' or ``irreversible.''

\subsubsection{Dissent Visibility}

Unlike anonymous voting, deliberative rounds make dissent \textit{visible}
(though not punitive):

\begin{quote}
``Shining the light on someone's disagreement within the consensus could help
the individual and the collective.''
\end{quote}

Visible dissent enables:
\begin{itemize}
    \item Identification of unaddressed concerns
    \item Opportunity for proposal modification
    \item Record of minority positions for future reference
    \item Accountability for officials who override consensus
\end{itemize}

\subsection{Participation History and Accountability}

The system maintains participation history for transparency and accountability,
without claiming to evaluate correctness.

\subsubsection{What Is Recorded}

For each participant:
\begin{itemize}
    \item Engagement verifications passed/failed (procedural record)
    \item Votes cast and their weights (participation record)
    \item Proposals submitted and their outcomes (contribution record)
    \item Coherence scores of votes participated in (process quality record)
\end{itemize}

\subsubsection{What Is Not Recorded or Evaluated}

The system does \textit{not}:
\begin{itemize}
    \item Label votes as ``correct'' or ``incorrect''
    \item Penalize voters for positions that differ from artifact conclusions
    \item Evaluate whether voters ``should have'' voted differently
    \item Assign semantic meaning to voting patterns
\end{itemize}

Participants are free to engage with evidence and reach their own conclusions.
The system verifies engagement, not agreement.

\subsubsection{Official Accountability}

For elected officials, participation history is public record:

\begin{itemize}
    \item Which engagement verifications they passed before voting
    \item How they voted on each proposal
    \item Whether they used override authority
    \item The coherence scores of decisions they participated in
\end{itemize}

This enables informed electoral choices without the system claiming to evaluate
whether official decisions were ``correct.''

\subsection{Node Reputation and Performance History}

The procedural infrastructure includes reputation tracking for \textit{infrastructure
nodes}, not just human participants.

\subsubsection{Oracle Node Reputation}

Nodes providing artifact references (document hashes, provenance data) accumulate
reputation based on:

\begin{itemize}
    \item Availability: Uptime and response latency
    \item Consistency: Variance in reported references
    \item Validity: Proportion of references that pass provenance checks
    \item Longevity: Duration of reliable service
\end{itemize}

This reputation feeds into test grid assembly: artifacts sourced through high-reputation
nodes have verified provenance chains.

\subsubsection{Validator Performance}

Consensus validators are tracked on:

\begin{itemize}
    \item Block production rate
    \item Missed slots
    \item Equivocation incidents
    \item Coherence score contributions
\end{itemize}

Poor-performing validators see reduced block rewards and eventual removal
from the active set.

\subsection{Evolution: EOS to Substrate to Quantum Harmony}

The procedural infrastructure evolved through three phases:

\subsubsection{Phase 1: EOS Smart Contract Prototype (2020--2021)}

Initial implementation as an EOS smart contract:
\begin{itemize}
    \item Basic test grid structure
    \item Token-curated artifact registration
    \item Simple pass/fail admissibility gates
    \item Proof-of-concept dynamic credentials
\end{itemize}

\subsubsection{Phase 2: Substrate Migration (2022--2023)}

Migration to Substrate framework:
\begin{itemize}
    \item Full pallet implementation
    \item NFT-based credential system
    \item Quadratic voting integration
    \item Multi-round deliberation support
\end{itemize}

\subsubsection{Phase 3: Quantum Harmony Integration (2024--2025)}

Current implementation with quantum enhancements:
\begin{itemize}
    \item Quantum entropy for vote weighting (Section~\ref{sec:governance-control})
    \item Post-quantum signatures for credential integrity
    \item Coherence-based consensus replacing stake-based finality
    \item QKD-protected artifact distribution
\end{itemize}

The core procedural architecture---test grids, dynamic credentials, quadratic
voting---persists across all phases. The quantum enhancements provide
cryptographic hardening without altering the democratic logic.

\subsection{Summary: The Procedural Stack}

The complete procedural infrastructure:

\begin{enumerate}
    \item \textbf{Artifact Layer}: Gatherers identify admissible artifacts, nodes verify provenance
    \item \textbf{Test Layer}: Curators assemble grids, token economics enforce quality
    \item \textbf{Credential Layer}: Dynamic NFTs track eligibility, life-sustaining
    requirements enforce engagement
    \item \textbf{Voting Layer}: Quadratic costs bound influence, reputation weights
    signal, entropy randomizes
    \item \textbf{Consensus Layer}: Coherence thresholds measure process quality,
    multi-round processes seek unanimity
    \item \textbf{Accountability Layer}: Participation recorded, officials tracked,
    history immutable
\end{enumerate}

This stack transforms voting from opinion aggregation into \textit{structured
preference revelation}---the foundation of augmented democracy.

\textbf{The system constrains how decisions are made, not what decisions must conclude.}
