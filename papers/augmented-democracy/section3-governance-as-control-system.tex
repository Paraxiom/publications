\section{Governance as a Control System}
\label{sec:governance-control}

This section presents a rigorous mapping from classical control theory to democratic governance,
grounded in a production implementation deployed on the Quantum Harmony blockchain. We demonstrate
that democratic processes can be formally modeled as coherence-constrained state machines, where
proposals represent state transitions, participants act as distributed sensors, and legitimacy
emerges from invariant preservation rather than mere majority arithmetic.

\subsection{The Control-Theoretic Frame}

We model a democratic system as a discrete-time control system $\mathcal{G} = (\mathcal{S}, \mathcal{U}, \mathcal{Y}, f, g, \mathcal{C})$ where:

\begin{itemize}
    \item $\mathcal{S}$ is the state space (governance configuration)
    \item $\mathcal{U}$ is the input space (proposals)
    \item $\mathcal{Y}$ is the output space (decisions)
    \item $f: \mathcal{S} \times \mathcal{U} \rightarrow \mathcal{S}$ is the state transition function
    \item $g: \mathcal{S} \rightarrow \mathcal{Y}$ is the output function
    \item $\mathcal{C} \subset \mathcal{S}$ is the coherence manifold---the set of legitimate states
\end{itemize}

The central requirement is \textit{coherence preservation}: for any valid transition $s_{t+1} = f(s_t, u_t)$,
if $s_t \in \mathcal{C}$, then $s_{t+1} \in \mathcal{C}$. Transitions that would exit the coherence
manifold are rejected.

\subsection{Proposals as State Transitions}

In the implementation, proposals are formalized as typed state transition requests with explicit
lifecycle semantics:

\begin{lstlisting}[language=Rust, caption={Proposal Status State Machine (lib.rs:114-124)}]
pub enum ProposalStatus {
    Submitted,        // s_0: Initial state
    UnderReview,      // s_1: Filtering phase
    ReadyForVoting,   // s_2: Cleared for deliberation
    VotingActive,     // s_3: Consensus formation
    VotingClosed,     // s_4: Measurement complete
    Approved,         // s_5+: On-manifold transition accepted
    Rejected,         // s_5-: Off-manifold transition blocked
    QuantumVerified,  // s_6: Cryptographic proof generated
}
\end{lstlisting}

This eight-state machine enforces a strict ordering: $s_0 \rightarrow s_1 \rightarrow s_2 \rightarrow
s_3 \rightarrow s_4 \rightarrow \{s_5^+, s_5^-\} \rightarrow s_6$. The system prohibits state
regression and enforces temporal guards:

\begin{lstlisting}[language=Rust, caption={Temporal Constraints on Voting Phase (lib.rs:430-435)}]
let current_block = <frame_system::Pallet<T>>::block_number();
ensure!(
    current_block >= proposal.voting_starts &&
    current_block <= proposal.voting_ends,
    Error::<T>::ProposalNotInVotingPhase
);
\end{lstlisting}

The \texttt{ReviewPeriod} and \texttt{VotingPeriod} configuration parameters define minimum dwell
times in each phase, preventing rushed transitions that could destabilize the system.

\subsection{Participants as Distributed Sensors}

Democratic participants are modeled as heterogeneous sensors with role-specific measurement capabilities:

\begin{lstlisting}[language=Rust, caption={Participant Roles as Sensor Types (lib.rs:88-94)}]
pub enum ParticipantRole {
    Submitter,   // Proposal origination sensor
    Voter,       // Binary/ternary measurement sensor
    Reviewer,    // Quality filtering sensor
    Validator,   // Consensus verification sensor
}
\end{lstlisting}

Each participant maintains state that influences their measurement weight:

\begin{lstlisting}[language=Rust, caption={Participant State Vector (lib.rs:77-86)}]
pub struct Participant<AccountId> {
    pub account: AccountId,
    pub role: ParticipantRole,
    pub registered_at: u64,
    pub contributions: u64,
    pub reputation: u64,           // Base control gain
    pub quantum_identity: Option<H256>,
}
\end{lstlisting}

The \texttt{reputation} field serves as the base \textit{control gain}---the amplification factor
that determines how strongly a participant's measurement influences the system output. Initial
reputation is set to 100 (lib.rs:336), establishing a baseline that can evolve through contribution history.

\subsection{Deliberation as Filtering}

The review period implements a low-pass filter on the proposal stream:

\begin{lstlisting}[language=Rust, caption={Review Period as Temporal Filter (lib.rs:392-393)}]
voting_starts: <frame_system::Pallet<T>>::block_number()
    + T::ReviewPeriod::get(),
voting_ends: <frame_system::Pallet<T>>::block_number()
    + T::ReviewPeriod::get() + T::VotingPeriod::get(),
\end{lstlisting}

This enforced delay serves multiple control-theoretic purposes:

\begin{enumerate}
    \item \textbf{Noise rejection}: Transient proposals (spam, emotional reactions) decay during the review period
    \item \textbf{Signal integration}: Reviewers accumulate information about proposal quality
    \item \textbf{Aliasing prevention}: The minimum period ensures adequate sampling of community response
\end{enumerate}

The system implements bounded queuing to prevent denial-of-service:

\begin{lstlisting}[language=Rust, caption={Bounded Proposal Queue (lib.rs:366-370)}]
let active_proposals = <Proposals<T>>::iter().count() as u32;
ensure!(
    active_proposals < T::MaxProposals::get(),
    Error::<T>::MaxProposalsReached
);
\end{lstlisting}

\subsection{Credentials as Control Gains}

Vote weight is computed as a function of reputation modulated by quantum entropy:

\begin{lstlisting}[language=Rust, caption={Control Gain Computation (lib.rs:617-631)}]
fn calculate_quantum_vote_weight(
    participant: &Participant<T::AccountId>,
    entropy: &[u8],
) -> Result<u64, Error<T>> {
    // Base weight from reputation
    let base_weight = participant.reputation;

    // Add quantum randomness factor (+/- 10%)
    let entropy_factor = (entropy[0] as u64 % 21) as i64 - 10;
    let quantum_adjustment =
        (base_weight as i64 * entropy_factor) / 100;

    let final_weight =
        (base_weight as i64 + quantum_adjustment).max(1) as u64;

    Ok(final_weight)
}
\end{lstlisting}

This formulation has precise control-theoretic semantics:

\begin{equation}
    w_i = \max\left(r_i + \frac{r_i \cdot (\eta_i \mod 21 - 10)}{100}, 1\right)
\end{equation}

where $w_i$ is the final weight, $r_i$ is the base reputation, and $\eta_i$ is the quantum entropy
byte. The entropy introduces a $\pm 10\%$ perturbation, which serves two purposes:

\begin{enumerate}
    \item \textbf{Sybil resistance}: An attacker controlling multiple identities cannot precisely
    predict aggregate weight
    \item \textbf{Tie-breaking}: Near-equal coalitions are resolved by entropy rather than timestamp
    or insertion order
\end{enumerate}

The \texttt{max(..., 1)} floor ensures no participant has zero influence---a stability constraint
preventing degenerate control configurations.

\subsection{Coherence as System Invariant}

The central contribution is the dual-condition consensus requirement:

\begin{lstlisting}[language=Rust, caption={Coherence-Constrained Consensus (lib.rs:666-667)}]
// Proposal approved if majority approves AND
// quantum confidence is high
let approved = approve_weight > reject_weight
    && quantum_confidence > 50;
\end{lstlisting}

This is not mere majority rule. Approval requires \textit{both}:

\begin{enumerate}
    \item $w_{\text{approve}} > w_{\text{reject}}$ (democratic condition)
    \item $\gamma > 50$ (coherence condition)
\end{enumerate}

where $\gamma$ is the \textit{quantum confidence score}, computed as:

\begin{lstlisting}[language=Rust, caption={Quantum Confidence Computation (lib.rs:679-715)}]
fn calculate_quantum_confidence(
    votes: &[QuantumVote<T::AccountId>],
) -> Result<u32, Error<T>> {
    // ... compute mean entropy across votes ...

    for vote in votes {
        let entropy_value = vote.quantum_nonce.iter()
            .map(|&b| b as f64)
            .sum::<f64>() / vote.quantum_nonce.len() as f64;

        entropy_variance +=
            (entropy_value - mean_entropy).powi(2);
    }

    entropy_variance /= votes.len() as f64;

    // Higher variance indicates better quantum randomness
    let confidence =
        ((entropy_variance / 255.0).min(1.0) * 100.0) as u32;

    Ok(confidence)
}
\end{lstlisting}

The coherence score measures the \textit{quality of the randomness distribution} across votes:

\begin{equation}
    \gamma = \min\left(\frac{\sigma^2_\eta}{255}, 1\right) \times 100
\end{equation}

where $\sigma^2_\eta$ is the variance of mean entropy values across all votes. This captures a
subtle but critical property: a legitimate vote should exhibit high entropy variance (true quantum
randomness), while a coordinated attack (replay, Sybil) will show artificially low variance.

\textbf{Critical clarification}: The coherence score $\gamma$ measures \textit{process quality},
not outcome correctness. A proposal can have high coherence and still be ``wrong'' by external
standards. A proposal can have low coherence despite being ``correct.'' The score measures whether
the \textit{process} exhibited properties consistent with legitimate deliberation---statistical
independence of participants and adequate entropy quality---independent of proposal content.

\textbf{Key insight}: The coherence threshold acts as a Lyapunov function. States with $\gamma \leq 50$
are \textit{outside the coherence manifold}---they may satisfy the democratic condition but fail the
process quality invariant. The system refuses to transition to such states, even under majority pressure.

\subsection{Rejection of Off-Manifold Transitions}

The error handling system explicitly enumerates forbidden transitions:

\begin{lstlisting}[language=Rust, caption={Off-Manifold Action Rejection (lib.rs:273-310)}]
pub enum Error<T> {
    ParticipantNotRegistered,    // Unidentified sensor
    ParticipantAlreadyRegistered,
    ProposalNotFound,            // Invalid reference
    ProposalNotInVotingPhase,    // Temporal violation
    AlreadyVoted,                // Replay attempt
    InsufficientQuantumEntropy,  // Entropy pool exhausted
    InvalidQuantumSignature,     // Authentication failure
    Unauthorized,                // Role violation
    MaxProposalsReached,         // Queue overflow
    MaxVotersReached,            // Capacity limit
    InvalidProposalData,         // Malformed input
    QuantumConsensusFailed,      // Coherence violation
}
\end{lstlisting}

Each error type corresponds to a specific invariant violation:

\begin{center}
\begin{tabular}{ll}
\toprule
\textbf{Error} & \textbf{Violated Invariant} \\
\midrule
\texttt{InsufficientQuantumEntropy} & Resource availability \\
\texttt{ProposalNotInVotingPhase} & Temporal ordering \\
\texttt{AlreadyVoted} & Idempotency \\
\texttt{Unauthorized} & Role-based access \\
\texttt{QuantumConsensusFailed} & Coherence threshold \\
\bottomrule
\end{tabular}
\end{center}

\subsection{Bounded Influence and Stability}

The system enforces hard bounds on participation:

\begin{lstlisting}[language=Rust, caption={Configuration Bounds (lib.rs:46-56)}]
#[pallet::constant]
type MaxProposals: Get<u32>;

#[pallet::constant]
type MaxVoters: Get<u32>;

#[pallet::constant]
type MaxReviewers: Get<u32>;
\end{lstlisting}

These bounds serve as \textit{stability constraints}:

\begin{itemize}
    \item \textbf{MaxProposals}: Prevents unbounded growth of the pending state space
    \item \textbf{MaxVoters}: Limits consensus computation complexity to $O(n)$
    \item \textbf{MinQuantumEntropy}: Ensures sufficient randomness reservoir
\end{itemize}

The entropy pool management (lib.rs:557-574) implements a FIFO queue with bounded capacity:

\begin{lstlisting}[language=Rust, caption={Entropy Pool Constraint (lib.rs:562-565)}]
ensure!(
    pool_size >= bytes &&
    pool_size >= T::MinQuantumEntropy::get(),
    Error::<T>::InsufficientQuantumEntropy
);
\end{lstlisting}

Operations fail gracefully when entropy is exhausted, preventing the system from entering a
low-coherence regime where randomness quality degrades.

\subsection{Cryptographic Proof Generation}

Each consensus produces a verifiable proof:

\begin{lstlisting}[language=Rust, caption={Consensus Proof Generation (lib.rs:718-741)}]
fn generate_quantum_consensus_proof(
    votes: &[QuantumVote<T::AccountId>],
    result: &QuantumConsensusResult,
) -> Result<BoundedVec<u8, ConstU32<1024>>, Error<T>> {
    let mut proof_data = Vec::new();

    // Aggregate vote signatures
    for vote in votes {
        proof_data.extend_from_slice(&vote.quantum_signature);
    }

    // Bind to result
    proof_data.extend_from_slice(&result.encode());

    // Add fresh entropy
    let quantum_entropy = Self::consume_quantum_entropy(64)?;
    proof_data.extend_from_slice(&quantum_entropy);

    // Commit
    let proof = BlakeTwo256::hash(&proof_data);

    Ok(proof.as_bytes().to_vec().try_into()?)
}
\end{lstlisting}

This proof commits to:
\begin{enumerate}
    \item All individual vote signatures (participation record)
    \item The computed result (outcome binding)
    \item Fresh quantum entropy (temporal uniqueness)
\end{enumerate}

The resulting Blake2-256 hash provides a succinct, collision-resistant commitment that can be
verified without reconstructing the full vote set.

\subsection{Summary: The Control Loop}

The complete governance control loop is:

\begin{enumerate}
    \item \textbf{Input}: Proposal $u_t$ submitted by registered Submitter
    \item \textbf{Filtering}: Review period applies temporal low-pass filter
    \item \textbf{Measurement}: Voters provide weighted observations $\{(v_i, w_i)\}$
    \item \textbf{Aggregation}: Compute $w_{\text{approve}}, w_{\text{reject}}, w_{\text{abstain}}$
    \item \textbf{Coherence Check}: Compute $\gamma$ from entropy distribution
    \item \textbf{Decision}: Accept transition iff $w_{\text{approve}} > w_{\text{reject}} \land \gamma > 50$
    \item \textbf{Proof}: Generate cryptographic commitment to decision
    \item \textbf{State Update}: Transition to $s_{t+1}$ or reject and remain at $s_t$
\end{enumerate}

\begin{figure}[h]
\centering
\begin{tikzpicture}[node distance=2cm, auto]
    \node[draw, rectangle] (input) {Proposal $u_t$};
    \node[draw, rectangle, right of=input, node distance=3cm] (filter) {Review Filter};
    \node[draw, rectangle, right of=filter, node distance=3cm] (sensors) {Voter Sensors};
    \node[draw, rectangle, below of=sensors] (aggregate) {Aggregation};
    \node[draw, diamond, below of=aggregate, aspect=2] (coherence) {$\gamma > 50$?};
    \node[draw, rectangle, left of=coherence, node distance=3cm] (accept) {Accept $s_{t+1}$};
    \node[draw, rectangle, right of=coherence, node distance=3cm] (reject) {Reject (stay $s_t$)};

    \draw[->] (input) -- (filter);
    \draw[->] (filter) -- (sensors);
    \draw[->] (sensors) -- (aggregate);
    \draw[->] (aggregate) -- (coherence);
    \draw[->] (coherence) -- node[above] {Yes} (accept);
    \draw[->] (coherence) -- node[above] {No} (reject);
\end{tikzpicture}
\caption{Governance Control Loop with Coherence Gate}
\label{fig:control-loop}
\end{figure}

The critical observation is that the coherence gate ($\gamma > 50$) is \textit{not optional}.
Even unanimous approval fails if the entropy distribution indicates manipulation. This inverts
the traditional democratic assumption: legitimacy is not derived from majority agreement, but
from the quality of the process that produced the agreement.

\subsection{Implications for Democratic Theory}

This control-theoretic framing has several implications:

\begin{enumerate}
    \item \textbf{Legitimacy is measurable}: The coherence score $\gamma$ provides a quantitative
    measure of process quality, independent of outcome.

    \item \textbf{Manipulation is detectable}: Coordinated attacks produce low-variance entropy
    distributions, triggering coherence failure before state corruption.

    \item \textbf{Stability is guaranteed}: Bounded queues, temporal guards, and entropy floors
    prevent the system from entering degenerate configurations.

    \item \textbf{Rollback is possible}: Rejected transitions leave the system at $s_t$, enabling
    retry with improved proposals rather than corrupted state recovery.
\end{enumerate}

The implementation demonstrates that coherence-constrained democratic systems are not merely
theoretical constructs but deployable infrastructure with formal guarantees.

\subsection{Scope of Control}

A final clarification on system scope:

\textbf{The governance system constrains how decisions are made, not what decisions must conclude.}

The control system ensures:
\begin{itemize}
    \item Participants have engaged with admissible artifacts (procedural gate)
    \item Influence is bounded and distributed (quadratic costs, reputation caps)
    \item The decision process exhibits statistical independence (coherence threshold)
    \item All transitions are auditable (cryptographic proofs)
\end{itemize}

The control system does \textit{not} ensure:
\begin{itemize}
    \item Outcomes are ``correct'' by any external standard
    \item Participants agree with artifact conclusions
    \item Proposals are wise, beneficial, or optimal
    \item The collective makes ``good'' decisions
\end{itemize}

This scope limitation is deliberate. A system that claimed to guarantee correct outcomes would
require semantic authority---the ability to evaluate truth---which is both philosophically
indefensible and practically capturable. By limiting scope to process quality, the system
remains defensible as infrastructure rather than oracle.
