\documentclass[11pt,a4paper]{article}
\usepackage{amsmath,amssymb,amsthm}
\usepackage{hyperref}
\usepackage{booktabs}

\newtheorem{theorem}{Theorem}
\newtheorem{lemma}[theorem]{Lemma}
\newtheorem{proposition}[theorem]{Proposition}
\newtheorem{corollary}[theorem]{Corollary}
\newtheorem{definition}{Definition}
\newtheorem{remark}{Remark}
\newtheorem{observation}{Observation}

\title{Topological Stasis Meets Spectral Gap:\\
Connecting Charlton's PACK Principle to\\
Toroidal Coherence Protection}

\author{Sylvain Cormier\\
Paraxiom Research\\
\texttt{sylvain@paraxiom.org}}

\date{February 2026}

\begin{document}

\maketitle

\begin{abstract}
We identify a structural correspondence between two independent results:
Charlton's PACK (Packing-Lock) principle for the Navier--Stokes
equations in $\mathbb{R}^3$ and the toroidal spectral gap bound for
quantum coherence protection on $T^2$.  Both derive dynamical
constraints---stasis in one case, exponential dephasing
suppression in the other---not from analytical control of the evolution
equations, but from global topological properties of the domain.  We
make the correspondence precise via a shared ``topological
inadmissibility'' structure: domain geometry forbids the dynamical
patterns that would produce pathological behavior.  This suggests a
broader principle: regularity, coherence, and stability are geometric
phenomena when the domain topology is sufficiently constrained.
\end{abstract}

\section{Introduction}

Two recent results, developed independently in different domains, share
a striking structural identity.

\textbf{Charlton (2026)} addresses the Clay Navier--Stokes Millennium
Problem by showing that in the infinite simply-connected plenum
$\mathbb{R}^3$, the incompressibility constraint $\nabla \cdot u = 0$
is not merely a local differential identity but a \emph{global geometric
constraint} that forces the acceleration field to vanish identically:
$a \equiv 0$ \cite{charlton2026pack,charlton2026ambiguity}.  The
mechanism is PACK (Packing-Lock): any strictly positive directional
acceleration on a set of nonzero measure forces a forward-half-space
momentum demand that diverges to infinity, contradicting finite forcing.

\textbf{Cormier (2026)} shows that the spectral gap
$\lambda_1 = 2 - 2\cos(2\pi/N)$ of the discrete Laplacian on the
Tonnetz torus $T^2$ bounds the decay rate of incoherent modes, providing
exponential dephasing suppression for quantum states and transformer
latent dynamics alike
\cite{cormier2026topological,cormier2026defensive}.  The mechanism:
high-frequency (noisy) components on the compact toroidal manifold decay
as $e^{-\lambda_1 t}$ while the zero mode (coherent signal) is
topologically protected.

Both results derive dynamical constraints from domain topology rather
than from analytical estimates on the evolution equations themselves.
This note makes the correspondence precise.


\section{The Shared Structure: Topological Inadmissibility}

\begin{definition}[Topological Inadmissibility]
Let $X$ be a domain, $\mathcal{E}$ an evolution equation on $X$, and
$\mathcal{P}$ a pathological behavior (e.g.\ blowup, decoherence,
drift).  We say $\mathcal{P}$ is \emph{topologically inadmissible} on
$X$ if the global topology of $X$ prohibits the dynamical patterns
required for $\mathcal{P}$, independent of initial conditions and the
specific form of $\mathcal{E}$.
\end{definition}

Both results instantiate this structure:

\begin{center}
\begin{tabular}{@{}lll@{}}
\toprule
& \textbf{PACK (Charlton)} & \textbf{Spectral Gap (Cormier)} \\
\midrule
Domain $X$ & $\mathbb{R}^3$ (infinite plenum) & $T^2$ (compact torus) \\
Constraint & $\nabla \cdot u = 0$ (incompressibility) & $\Delta_{T^2}$ has gap $\lambda_1 > 0$ \\
Pathology $\mathcal{P}$ & Finite-time blowup & Decoherence / drift \\
Mechanism & Infinite forward demand & Exponential mode decay \\
Result & $a \equiv 0$ (stasis) & $\|\delta(t)\| \leq e^{-\lambda_1 t}$ \\
Character & Topological, not analytical & Geometric, not perturbative \\
\bottomrule
\end{tabular}
\end{center}

The key parallel: in both cases, the topology of the domain itself does
the work.  No energy estimates, no Sobolev embeddings, no Gr\"onwall
inequalities.  The geometry \emph{forbids} the pathological dynamics.


\section{PACK as an Infinite-Domain Spectral Gap}

We can sharpen the analogy.  On the compact torus $T^2$, the Laplacian
has discrete spectrum $\{0, \lambda_1, \lambda_2, \ldots\}$ with
$\lambda_1 > 0$.  The spectral gap separates the coherent zero mode
from all noisy modes, and the Poincar\'e inequality gives:

\begin{equation}\label{eq:poincare}
\|f - \bar{f}\|^2 \leq \frac{1}{\lambda_1} \|\nabla f\|^2
\end{equation}

This bounds fluctuations in terms of gradients.  On $T^2$, the gap is
finite and positive, giving exponential decay of perturbations.

On $\mathbb{R}^3$, the Laplacian has \emph{continuous} spectrum starting
at zero---no spectral gap in the classical sense.  But PACK provides a
different kind of gap: not spectral but \emph{topological}.  The
incompressibility constraint $\nabla \cdot u = 0$ in an infinite
simply-connected domain forces the projection onto divergence-free
fields to be so rigid that the ``effective gap'' is infinite:

\begin{observation}
PACK can be interpreted as $\lambda_{\mathrm{eff}} = \infty$ for the
incompressibility constraint on $\mathbb{R}^3$.  While $T^2$ gives
finite exponential suppression ($e^{-\lambda_1 t}$), the infinite
plenum gives total suppression ($a \equiv 0$).
\end{observation}

This places the two results on a continuum:

\begin{center}
\begin{tabular}{@{}lccc@{}}
\toprule
\textbf{Domain} & \textbf{Spectral gap} & \textbf{Suppression} & \textbf{Character} \\
\midrule
$\mathbb{R}^n$ (unconstrained) & $\lambda_1 = 0$ & None & Drift/blowup possible \\
$T^2$ (compact torus) & $\lambda_1 > 0$ (finite) & Exponential & Coherence protection \\
$\mathbb{R}^3$ (incompressible plenum) & $\lambda_{\mathrm{eff}} = \infty$ & Total & Topological stasis \\
\bottomrule
\end{tabular}
\end{center}

The hierarchy $0 < \lambda_1 < \infty$ corresponds to the hierarchy of
protection: no protection, exponential protection, absolute protection.


\section{Geometric Thermodynamics Interpretation}

McGinty (2026) frames thermodynamic systems as Riemannian manifolds
where curvature encodes interaction strength and phase transitions appear
as curvature singularities \cite{mcginty2026geometric}.  This provides
the unifying language:

\begin{itemize}
\item \textbf{Curvature $=$ constraint strength.}  On $T^2$, the
  constant positive curvature of the spectral gap manifold provides
  uniform dephasing suppression.  On $\mathbb{R}^3$ with
  incompressibility, the effective curvature of the constraint manifold
  is infinite---so rigid that no acceleration is admissible.

\item \textbf{Phase transitions $=$ curvature divergences.}  In the
  optomechanical systems of \cite{cormier2026defensive}, the
  cooperativity parameter $C$ acts as a curvature parameter.  When
  $C \to \infty$, the bare-state description undergoes a geometric phase
  transition to the dressed-state regime.  Charlton's PACK is the
  fluid-dynamical analog: incompressibility is so strong that the
  velocity-field manifold admits no non-trivial flows.

\item \textbf{Stability $=$ positive-definite metric.}  The toric code
  threshold ($p < p_c \approx 0.09$) requires the Tonnetz metric to
  remain positive-definite.  PACK requires the momentum-balance metric
  on $\mathbb{R}^3$ to remain finite.  In both cases, metric
  degeneration corresponds to loss of protection.
\end{itemize}


\section{Implications}

\subsection{For Quantum Coherence}

The PACK principle suggests that stronger topological constraints yield
stronger coherence protection.  Our current model uses $T^2$ (finite
spectral gap, exponential suppression).  Embedding the torus in a
higher-dimensional incompressible structure could provide additional
protection channels---a ``PACK shield'' around the quantum state.

Concretely: nanotori bundle geometries (ring-of-rings, Hopf links)
already improve cooperative Q by factors of $4$--$6\times$ via
collective phase matching.  These bundles increase the effective
topological rigidity of the system, moving along the continuum toward
stronger suppression.

\subsection{For Navier--Stokes}

The spectral gap perspective suggests a quantitative refinement of PACK:
rather than total stasis ($a \equiv 0$), one might characterize the
\emph{rate} at which near-incompressible perturbations are suppressed as
a function of the domain's topological complexity (genus, connectivity,
boundary conditions).  This would connect to the well-known result that
Navier--Stokes regularity is easier to establish on $T^3$ (periodic
domain) than on $\mathbb{R}^3$---precisely because $T^3$ has a finite
spectral gap.

\subsection{For Machine Learning}

The toroidal logit bias \cite{cormier2026topological} already
demonstrates that imposing $T^2$ geometry on transformer latent dynamics
reduces hallucination by $+2.8$pp on TruthfulQA across four models.
The PACK analogy suggests that even stronger geometric constraints
(e.g.\ volume-preserving flows in latent space, divergence-free
attention) could yield correspondingly stronger coherence guarantees.


\section{Conclusion}

Charlton's PACK and Paraxiom's spectral gap bound are manifestations of
a single principle: \textbf{domain topology constrains dynamics}.
Regularity in fluid mechanics, coherence in quantum systems, and
stability in neural networks are not achieved by analytical control of
the evolution equations but by geometric properties of the space in
which evolution occurs.

The two results sit on a continuum parameterized by the effective
spectral gap: $\lambda_1 = 0$ (no protection, pathology possible),
$0 < \lambda_1 < \infty$ (exponential protection, toroidal coherence),
$\lambda_{\mathrm{eff}} = \infty$ (total protection, topological
stasis).  The unifying thesis is that \emph{geometry is not merely
descriptive---it is prescriptive}.  The shape of the domain determines
what dynamics are admissible.

\begin{thebibliography}{10}

\bibitem{charlton2026pack}
J.~P. Charlton~Jr.
\newblock Acceleration as Topology: The Inertial Lock of the Infinite
  Simply-Connected Plenum.
\newblock Zenodo, 2026.
\newblock \href{https://doi.org/10.5281/zenodo.18630611}{DOI:~10.5281/zenodo.18630611}.

\bibitem{charlton2026ambiguity}
J.~P. Charlton~Jr.
\newblock Enabling Cross-Silo Innovation for the CPNS: A Public Specification
  Resolving Ambiguity.
\newblock Zenodo, 2026.
\newblock \href{https://doi.org/10.5281/zenodo.18548206}{DOI:~10.5281/zenodo.18548206}.

\bibitem{cormier2026topological}
S.~Cormier.
\newblock Topological Constraints for Coherent Language Models: Why Geometry
  Prevents Hallucination.
\newblock Zenodo, 2026.
\newblock \href{https://doi.org/10.5281/zenodo.18624950}{DOI:~10.5281/zenodo.18624950}.

\bibitem{cormier2026defensive}
S.~Cormier.
\newblock Toroidal Coherence Architecture: Defensive Technical Disclosure.
\newblock Zenodo, 2026.
\newblock \href{https://doi.org/10.5281/zenodo.18595753}{DOI:~10.5281/zenodo.18595753}.

\bibitem{mcginty2026geometric}
C.~McGinty.
\newblock Geometric Thermodynamics: Curvature, Entropy, and the Geometry of
  Physical Reality.
\newblock Plenum of Energy Prophecies (LinkedIn Newsletter), February 2026.

\bibitem{fefferman2000}
C.~Fefferman.
\newblock Existence and Smoothness of the Navier--Stokes Equation.
\newblock Clay Mathematics Institute Millennium Problem Description, 2000.

\end{thebibliography}

\end{document}
